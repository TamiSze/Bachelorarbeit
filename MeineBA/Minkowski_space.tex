\subsection{Minkowski space \checkmark}
	First we need to know what a metric is. 
	The mathematical definition is as follows[see differentialgeometrie.pdf p.85]
	A metric
	$ d:X\times X \rightarrow \R$ is a function, that satisfies the conditions:
		\begin{enumerate}[(i)]
			\item $d(x,y)=d(y,x)$;
			\item $d(x,y)\geq 0$, with equality if and only if $x=y$;
			\item $d(x,y)+d(x,z)\geq d(x,z)$
		\end{enumerate}
	for any $x,y,z \in X$.
	 
	But in physics, we express the metric as an invariant line element:
		\begin{equation}
		 	\diff s^2 = g_{\mu\nu} \diff x^\mu \diff x^\nu
		\end{equation}
	so for getting $d(x,y)$ one have to take the square root of $\diff s^2$ and integrate. $g_{\mu\nu}(x)$ is the \textit{metric tensor}.
	 
	 For the of ordinary Minkowski space the metric tensor is now $g_{\mu\nu}(x) = \eta_{\mu\nu}$ so that
		\begin{equation}
			\diff s^2=- \diff t^2 + \diff x^2 + \diff y^2 + \diff z^2 = \eta_{\mu\nu} \diff x^\mu \diff x^\nu
		\end{equation}
	Where $\eta_{\mu\nu}$ is of the form

		\begin{equation}
		    \left(\eta_{\mu\nu}\right)=
		    \left(\begin{array}{cccc}
		    	-1 & 0 & 0 & 0\\
		    	0 & +1 & 0 & 0\\
		    	0 & 0 & +1 & 0\\
		    	0 & 0 & 0 & +1
			\end{array}\right)
%	    \begin{pmatrix}
%	    	-1 & 0 & 0 & 0\\
%	    	0 & 1 & 0 & 0\\
%	    	0 & 0 & 1 & 0\\
%	    	0 & 0 & 0 & 1
%	    \end{pmatrix}
		\end{equation}
	 
%%% Local Variables: 
%%% TeX-master: "main.tex" 
%%% End: 

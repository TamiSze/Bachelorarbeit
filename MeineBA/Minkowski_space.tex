\subsection{Minkowski space \checkmark}
	First we need to know what a metric is. 
	The mathematical definition is as follows[see differentialgeometrie.pdf p.85] \todo{why not a cite{?}}
	A metric
	$ d:X\times X \rightarrow \R$ is a function, that satisfies the conditions:
		\begin{enumerate}[(i)]
			\item $d(x,y)=d(y,x)$;
			\item $d(x,y)\geq 0$, with equality if and only if $x=y$;
			\item $d(x,y)+d(x,z)\geq d(x,z)$
		\end{enumerate}
	for any $x,y,z \in X$.
	 
	But in physics, we express the metric as an invariant line element:
		\begin{equation}
		 	\d s^2 = g_{\mu\nu} \d x^\mu \d x^\nu
		\end{equation}
		\todo{dies funktioniert nur wenn du eine riemannische metrik hast, aber für den lorentzfall liefert das keine metrik im	obigen sinne, denn das wurzel ziehen liefert hier keine reelle zahl und greift die obige definition nicht mehr, auch ein übergang ist komplexe kann hier nicht funktionieren, denn dort würden ii) und iii) keinen sinn mehr ergeben da sich die komplexen zahlen nicht anordnen lassen}
			so for getting $d(x,y)$ one have to take the square root of $\d s^2$ and integrate. $g_{\mu\nu}(x)$ is the \textit{metric tensor}.
	 
	 For the ordinary Minkowski space the metric tensor is now $g_{\mu\nu}(x) = \eta_{\mu\nu}$
	 so that
		\begin{equation}
			\d s^2=- \d t^2 + \d x^2 + \d y^2 + \d z^2 = \eta_{\mu\nu} \d x^\mu \d x^\nu.
		\end{equation}
	Where $\eta_{\mu\nu}$ is of the form

		\begin{equation}
		    \left(\eta_{\mu\nu}\right)=
		    \left(\begin{array}{cccc}
		    	-1 & 0 & 0 & 0\\
		    	0 & +1 & 0 & 0\\
		    	0 & 0 & +1 & 0\\
		    	0 & 0 & 0 & +1
			\end{array}\right)
%	    \begin{pmatrix}
%	    	-1 & 0 & 0 & 0\\
%	    	0 & 1 & 0 & 0\\
%	    	0 & 0 & 1 & 0\\
%	    	0 & 0 & 0 & 1
%	    \end{pmatrix}
		\end{equation}
	 
%%% Local Variables: 
%%% TeX-master: "main.tex" 
%%% End: 

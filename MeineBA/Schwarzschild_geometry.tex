\subsection{Schwarzschild geometry \checkmark}
	
	The Schwarzschild geometry is a source-free solution of Einstein's equation \marginpar{[see chapter21 in ART Fließbach]} with spherical symmetry. The latter means, the solution is invariant under rotations. 
	At large distances it approaches the ordinary Minkowski space.
	The spacetime metric of the Schwarzschild geometry looks like this:\\
		\begin{equation} \label{p.3 (2.1)}
		\diff s^2=-\frac{r-2GM}{r}~ \diff t^2+\frac{r}{r-2GM}~
		\diff r^2+r^2 \left( \diff \theta^2+\sin^2\theta \diff \phi^2 \right).
		\end{equation}
	The G is Newton's gravitational constant, M is a mass parameter, which comes from idealisation if one is looking at the black hole from a distance $r\gg 2GM$.
	The term in the brackets is often shorten by $\diff \Omega_2^2$ which is on the two sphere $\Sp^2$.\footnote{A sphere is a n-dimensional manifold in Euclidean $(n+1)$~-~dimensional space.}
	
	The most interesting radii are $r=0$ and $r=2GM$.
	At $r=0$ we have a singularity, i.e. the sphere $\Sp^2$ goes to zero size and the Schwarzschild metric diverges.
	This can be described by the Riemann tensor $R_{\alpha\beta\gamma \delta}R^{\alpha\beta\gamma\delta}$ which encodes the tidal effects\footnote{tidal effects: The nearer an object is to a black hole, the more deformed it becomes because of the gravitational force. In the end, it will be destroyed before it reaches the singularity, except in Planck-scale physics.} on free-falling objects.
	
	$r_{s}\equiv 2GM$ is called the \textit{Schwarzschild radius}. At this radius, the metric has a singularity too, but that is just because of our choice of coordinates. Here the signs of $\diff r^2$ and $\diff t^2$ switch, so the coordinate $r$ becomes timelike, and the coordinate $t$ becomes spacelike. That causes that everything under $r_{s}$ will inevitably fall into the singularity. 
	So nothing, even massless particles like light, cannot move forward in ordinary time. This means for an observer in $r>r_{s}$, everything in $r<r_{s}$ is invisible and inversely. 
	
	That is, why $r_{s}$ is often called the \textit{event horizon} or just \textit{horizon}.
	In addition, the closer someone is to $r_{s}$ while sending a signal, the lower will be its energy when it reaches $r\gg 2GM$. This phenomenon is called \textit{gravitational redshift}.

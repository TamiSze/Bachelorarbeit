\subsection{Entanglement in the Rindler decomposition \checkmark}
	What happens, if we want to cross the $x=0$ surface? In order to find that out, we put the system in a mixed state with
		\begin{equation} \label{mixed_state_firewall}
			\rho=\rho_L \otimes \rho_R
		\end{equation}
	instead of having a ground state $|\Omega\rangle$. Here $\rho_L$ and $\rho_R$ are the thermal density matrices, which we get if we trace out the respectively other one in the vacuum $|\Omega\rangle$.
	If the fields are completely discontinuous like in \eqref{mixed_state_firewall} , the gradient term of the Hamiltonian will diverge at $x=0$. If you are an observer in the left or right Rindlers wedge it seems, that you just have vacuum state, but the energy is infinite. 
	
	Its typical field fluctuation is given by $\frac{1}{\epsilon}$ where $\epsilon$ is a short-distance length cutoff. \marginpar{was ist damit genau gemeint?} So it is valid that
		\begin{equation}
			\partial_x \phi|_{x=0} \propto \frac{1}{\epsilon^2}.
		\end{equation}
	Which means, that the gradient term in the Hamiltonian contributes
		\begin{equation}
			\diff x \int \diff^2 y (\partial_x\phi)^2 \propto 
			\epsilon \int \diff^2 y \frac{1}{\epsilon^4} 
			= \frac{A}{\epsilon^3}
		\end{equation}	
	The smaller $\epsilon$ is the bigger becomes the energy and $\epsilon$ is even to the third power.
	 			
	This is called a \textit{firewall}: A huge concentration of energy at $x=0$, that annihilates anybody who tries to jump through the Rindler horizon into the future wedge.
	
	For example the product states $|00\rangle$ and $|11\rangle$ of the states $\frac{1}{\sqrt{2}}
	(|00\rangle \pm |11\rangle)$ which shall both have smooth horizons, should have them too, because of linearity of quantum mechanics. But as we just saw, no product state possibly can have a smooth horizon in QFT. So, for going smoothly through the Rindlers horizon we need not only any entanglement but it must have the \textit{right entanglement} too.
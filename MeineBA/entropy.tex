\subsection{Entropy and thermodynamics \checkmark}
	If a black hole has a temperature and an energy, it must also have an entropy. So let's remember the inner energy in statistical mechanics\footnote{$\diff E = 
	T \diff S - p \diff V + \mu \diff N$} and derive it to:
		\begin{equation}
			\frac{\diff S}{\diff E} = \left. \frac{1}{T} \right|_{V,N=const.}
		\end{equation}
	For the black hole we have $T = \frac{1}{8 \pi G M}$ and $M=E$. If we assume that $S(E=0) = 0$, we can write:
		\begin{align}
			\frac{\diff S}{\diff M} &= 8 \pi G M \nonumber \\
			&\Leftrightarrow \int_0^S \diff S' = \int_0^M 8 \pi M' \diff M' \nonumber\\
			&\Leftrightarrow S = 4 \pi G M^2 & & \Bigm| r_s = 2GM \nonumber\\
			&\Leftrightarrow S = \frac{r_s^2 \pi}{G}  & & \Bigm| A=4 \pi r_s^2 \text{~and~} l_p = \sqrt{8 \pi G} \nonumber\\
			&\Leftrightarrow~
			S = \frac{A}{4G} = 2 \pi \frac{A}{l_p} \label{entropy}
		\end{align}
	For a black hole of the mass of our sun, this entropy would be $10^{78}$ which is enormous! If we take the sun like it is, the entropy would ``just'' be $10^{60}$. 
	
	Historical the entropy of a black hole was discovered before its temperature. With help of classical general relativity, we can see that the area of an event horizon of a black hole never decreases which looks quite like the second law of thermodynamics. Together with certain formal definition of the entropy, where it is proportional to the horizon area and a temperature indirect proportional to the Schwarzschild radius, the first law of thermodynamics with $\diff M = T \diff S$ is satisfied, too. 
	
	Jacob Bekenstein was holding out that this entropy should be that kind of statistical entropy of a black hole, that counts the number of ways it could have formed itself. 
	In a thought experiement he was throwing some systems with own entropy into a black hole and discovered that the interior entropy was growing faster, than the exterior entropy was sinking because of the systems loss. 
	So this means, that the entropy must be given by some constant proportional to the horizon area in Planck units.
	Bekenstein called this the \textit{Genereralized Second Law}.
	
	As Hawking published his paper about the temperature of a black hole, Bekensteins theory strongly reinforced. This is why the entropy of a black hole is often called \textbf{Bekenstein-Hawking entropy}. 
	
	In \textit{string theory} this idea of an entropy counting microstates is strong supported. For example in many situations where we count the states of a long vibrating string we can see how big the entropy of a black hole is. In some supersymmetric cases it is even possible to compute the $\frac{1}{4}$ in equation \eqref{entropy}.
\subsection{Evaporation N}
	Until now the black hole could radiate an infinit amount of energy. This is of course not physically correct. We will now resore the dynamical gravity so that the mass becomes time independent. This means, it is no longer reasonable to set the Schwarzschild radius to one. From now on $r_s= 2GM$.	
	
	So the old definition of the black hole's energy was that is generated translations of $t$ in the Schwarzschild geometry. But in the general relativity it doesn't work any more to attach this definition to the coordinates.
	
	In an asymtotical flat space and using Hamilton formalism while the general relativity is inlcuded, the energy is an integration over the boundary at the two sphere at $r\rightarrow \infty$. In the Penrose diagram, this would be the $i^0$ boundary. It is also called the ADM energy\footnote{cherished to Arnowitt, Deser and Misner} and is not only conserved but for a black hole with mass $M$, the ADM energy is also $M$, if it was formed form a collapsing shell. 
	
	Now we might be able to calculate how long a black hole lives.
	First we need the total energy disposal:
	\begin{align}
		\frac{\diff E}{\diff t} =
		\sum_{\ell, m} \int_{0}^{\infty} \frac{\diff \omega}{2 \pi}
		\frac{\omega P_{abs}(\omega,\ell)}{e^{\beta \omega} - 1}
	\end{align}
			
	%hawkings paper [2]

	%\clearpage
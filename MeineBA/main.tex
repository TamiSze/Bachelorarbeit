\documentclass[english, a4, 12pt]{scrartcl}
\usepackage{babel}
\usepackage[T1]{fontenc}
\usepackage[utf8]{inputenc}
\usepackage{lmodern}

\usepackage{amsmath}
\usepackage{amsfonts}
\usepackage{dsfont}
\usepackage{enumerate}
\usepackage{graphicx}
\usepackage{mathrsfs}
\usepackage{placeins}
\usepackage{braket}
\usepackage{color}
\usepackage[colorlinks=true, linkcolor=cyan]{hyperref}

%for better todos and critic
\usepackage{todonotes}

\usepackage[singlelinecheck=off]{caption}
\setkomafont{captionlabel}{\bfseries}
\renewcommand*{\captionformat}{.~~}
\setcapindent{0pt}

\usepackage[backend=bibtex, style=numeric-comp, sorting=nty]{biblatex}
\bibliography{bib}
\definecolor{forestgreen}{rgb}{0.0, 0.5, 0.0}


%\usepackage[samepage]{footmisc}
%\usepackage{jheppub}
%\usepackage{verbatim}
%\usepackage{showlabels}
%\pdfoutput=1
%\usepackage{blindtext}

\newcommand{\R}{\mathds{R}}
\newcommand{\Sp}{\mathbb{S}}
\newcommand{\Hil}{\mathcal{H}}
\newcommand{\diff}{\mathrm{d}}
%better and more intuitive
\renewcommand{\d}{\operatorname{d}}	
%good notation for statees bracket, define bra and ket as markos
\renewcommand{\bra}[1]{| #1 \rangle}
\renewcommand{\ket}[1]{\langle #1 |}

%hilfmittelende

\begin{document}
\numberwithin{equation}{section}

	\begin{titlepage}
		\begin{minipage}[c][\textheight][c]{\textwidth}
			\begin{center}
				{ \Huge \textbf{Bachelorthesis} }
				
				\vspace*{1cm}
				{\Large Tamara Szecsey}
				
				\vspace*{1cm}
				{\Large \today}
				
				\vspace*{4cm}
				\hspace*{1cm} \includegraphics[height=30ex]{LOGO_UR}
			\end{center}
		\end{minipage}
	\end{titlepage}
	
\thispagestyle{empty}
\tableofcontents	
\clearpage	
\setcounter{page}{1}
\section{Introduction}
	This work is up to the basics of theoretical examination of black holes. Even for non-physicists, these astronomical objects are the most interessing and fascinating ones and despite other subjects in physics, everyone heard of it before in the popular science.
	
	So my task was to rewrite parts of Jerusalem Lectures on Black Holes and Quantum Information of Daniel Harlow which you can have a look onto here \cite{JerusalemLectures}. Most time I didn't mark the pictures or formulas I took from Daniel Harlows lectures, because it is clear, that nearly everything was his work. I tried to make it possible for a bachelor student to understand this part of the lecture by introducing more basics, but this subject is still very hard without having heard a QFT lecture. To my regret this part of theoretical physics is still not experimentally verified because black holes are difficult to monitor especially near it. 
	
	The thesis beginns with the introduction of different spaces and geometries that were found by solving the Einstein equations. After that we will establish new parametrizations like the tornoise coorinates in order to include the whole spacetime into a diagram, which will be called Penrose diagrams and will haunt us through every theory about black holes. 
	Then I give a short introduction about quantum field theory and entanglement which we need to know for  the rindler space and what Hawking radiation is. In the end we will have a look at the problem, that information can't just be removed by black holes.
	
	\subsection{Conventions}
	The whole time I will use of course the natural units $c = \hbar = 1$. In addition in the most chapters we will set the Schwarzschildradius $r_s =2GM$ to one for simplicity. This will be retracted when we will discuss the black holes mass as time-dependent. 
	
\section{Basic environments}
	
	%\subsection{Minkowski space}
	\subsection{Minkowski space \checkmark}
	First we need to know what a metric is. 
	The mathematical definition is as follows[see differentialgeometrie.pdf p.85] \todo{why not a cite{?}}
	A metric
	$ d:X\times X \rightarrow \R$ is a function, that satisfies the conditions:
		\begin{enumerate}[(i)]
			\item $d(x,y)=d(y,x)$;
			\item $d(x,y)\geq 0$, with equality if and only if $x=y$;
			\item $d(x,y)+d(x,z)\geq d(x,z)$
		\end{enumerate}
	for any $x,y,z \in X$.
	 
	But in physics, we express the metric as an invariant line element:
		\begin{equation}
		 	\d s^2 = g_{\mu\nu} \d x^\mu \d x^\nu
		\end{equation}
		\todo{dies funktioniert nur wenn du eine riemannische metrik hast, aber für den lorentzfall liefert das keine metrik im	obigen sinne, denn das wurzel ziehen liefert hier keine reelle zahl und greift die obige definition nicht mehr, auch ein übergang ist komplexe kann hier nicht funktionieren, denn dort würden ii) und iii) keinen sinn mehr ergeben da sich die komplexen zahlen nicht anordnen lassen}
			so for getting $d(x,y)$ one have to take the square root of $\d s^2$ and integrate. $g_{\mu\nu}(x)$ is the \textit{metric tensor}.
	 
	 For the ordinary Minkowski space the metric tensor is now $g_{\mu\nu}(x) = \eta_{\mu\nu}$
	 so that
		\begin{equation}
			\d s^2=- \d t^2 + \d x^2 + \d y^2 + \d z^2 = \eta_{\mu\nu} \d x^\mu \d x^\nu.
		\end{equation}
	Where $\eta_{\mu\nu}$ is of the form

		\begin{equation}
		    \left(\eta_{\mu\nu}\right)=
		    \left(\begin{array}{cccc}
		    	-1 & 0 & 0 & 0\\
		    	0 & +1 & 0 & 0\\
		    	0 & 0 & +1 & 0\\
		    	0 & 0 & 0 & +1
			\end{array}\right)
%	    \begin{pmatrix}
%	    	-1 & 0 & 0 & 0\\
%	    	0 & 1 & 0 & 0\\
%	    	0 & 0 & 1 & 0\\
%	    	0 & 0 & 0 & 1
%	    \end{pmatrix}
		\end{equation}
	 
%%% Local Variables: 
%%% TeX-master: "main.tex" 
%%% End: 
 

	%\subsection{de Sitter space}
	\subsection{De Sitter space \checkmark} 
	The de Sitter space is a good approximation of the geometry of our universe today and in the past while inflation\footnote{inflation: a well proofed theory of exponential expansion of space and mass in the early universe}. It is a solution of the Einstein's equations with positive energy and is a submanifold of the Minkowski space.
	
%	For defining a four-dimensional de Sitter space, we have a $(4+1)$-dimensional Minkowski space, where it would be a hyperboloid with \marginpar{siehe \cite{stringtheory_revolution} S.133}
%		\begin{equation}
%			\sum_{i=1}^4 (x^i)^2 - (x^0)^2 = R^2.			
%		\end{equation}
	
	In order to understand the form of the De Sitter metric, let's start with the Riemann tensor\footnote{It should be known, that in general relativity we are in a four-dimensional Riemann manifold, where the Riemann tensor gives us the curvature.} for a maximally symmetric n-dimensional manifold: 
		\begin{align}
			R_{\mu \nu \lambda \xi} = \kappa (g_{\mu \lambda} g_{\nu \xi} - g_{\mu \xi} g_{\nu \lambda})
		\end{align}
	Here $\kappa$ is a measure of the curvature.\footnote{It is normalized and more specifically, it's the Ricci curvature, but this and more information about $\kappa$ is not needed for understanding the de Sitter space.}
	
	The already known Minkowski space is a maximally symmetric spacetime with a vanishing curvature $\kappa = 0$. If the curvature is positive, which means $\kappa > 0$, then it is called de Sitter space. For building its metric, we first take the five-dimensional Minkowski space:
		\begin{align}
			\diff s^2_5 = -\diff u^2 + \diff x^2 + \diff y^2 + \diff z^2 + \diff w^2
		\end{align}
	And a corresponding hyperboloid:
		\begin{align}
			-u^2 + x^2 + y^2 + z^2 + w^2 = 1
		\end{align}
	We use new coordinates $\tau, \chi, \theta$ and $\phi$:
		\begin{align}
			\begin{split}
			u &= \sinh (\tau) \\
			w &= \cosh (\tau) \cos \chi \\
			x &= \cosh (\tau) \sin \chi \cos \theta \\
			y &= \cosh (\tau) \sin \chi \sin \theta \cos \phi \\
			z &= \cosh (\tau) \sin \chi \sin \theta \sin \phi
			\end{split}
		\end{align}
	So in the end the metric on the hyperboloid looks like:
		\begin{align} \label{de-sitter}
			ds^2 = -\diff \tau^2 + \cosh^2(\tau) \, \diff \Omega^2_3 
		\end{align}
	with $\diff \Omega_3^2 = \diff \chi^2 + \sin^2 \chi (\diff \theta^2 + \sin^2 \theta \diff \phi^2)$
	which is the metric on a three-sphere $\Sp^3$.\footnote{A sphere is a n-dimensional manifold in Euclidean $(n+1)$~-~dimensional space.}
	
	Just for completion, a maximally symmetric spacetime with negative curvature $\kappa < 0$ is called anti-de Sitter space (further reading: \cite{SpacetimeCarroll}).
	 	
	%\subsection{Schwarzschild geometry}
	\subsection{Schwarzschild geometry \checkmark}
	
	The Schwarzschild geometry is a source-free solution of Einstein's equation \marginpar{[see chapter21 in \cite{ARTfliesbach}]} with spherical symmetry. The latter means, the solution is invariant under rotations. 
	At large distances it approaches the ordinary Minkowski space.
	The spacetime metric of the Schwarzschild geometry looks like this:\\
		\begin{equation} \label{p.3 (2.1)}
		\diff s^2=-\frac{r-2GM}{r}~ \diff t^2+\frac{r}{r-2GM}~
		\diff r^2+r^2 \left( \diff \theta^2+\sin^2\theta \diff \phi^2 \right).
		\end{equation}
	The G is Newton's gravitational constant, M is a mass parameter, which comes from idealization if one is looking at the black hole from a distance $r\gg 2GM$.
	The term in the brackets is often shorten by $\diff \Omega_2^2$ which is the metric on the two-sphere $\Sp^2$.
	
	The most interesting radii are $r=0$ and $r=2GM$.
	At $r=0$ we have a singularity, i.e. the sphere $\Sp^2$ goes to zero size and the Schwarzschild metric diverges.
	This can be described by the Riemann tensor $R_{\alpha\beta\gamma \delta}R^{\alpha\beta\gamma\delta}$ which encodes the tidal effects\footnote{tidal effects: The nearer an object is to a black hole, the more deformed it becomes because of the gravitational force. In the end, it will be destroyed before it reaches the singularity, except in Planck-scale physics.} on free-falling objects.
	
	$r_{s}\equiv 2GM$ is called the \textit{Schwarzschild radius}. At this radius, the metric has a singularity too, but that is just because of our choice of coordinates. Here the signs of $\diff r^2$ and $\diff t^2$ switch, so the coordinate $r$ becomes timelike, and the coordinate $t$ becomes spacelike. That causes that everything under $r_{s}$ will inevitably fall into the singularity. 
	So nothing, even massless particles like light, cannot move forward in ordinary time. This means for an observer in $r>r_{s}$, everything in $r<r_{s}$ is invisible and inversely. 
	
	That is, why $r_{s}$ is often called the \textit{event horizon} or just \textit{horizon}.
	In addition, the closer someone is to $r_{s}$ while sending a signal, the lower will be its energy when it reaches $r\gg 2GM$. This phenomenon is called \textit{gravitational redshift}.

				
\section{Basic ways of visualising the black hole problems on paper}
	
	%subsection{The Kruskal extenstion}
	\subsection{The Kruskal extension \textbf{N}}
		\begin{figure}[htbp]
			\begin{center}
				\includegraphics[scale=1]{kruskal}
			\end{center}
			\caption{This is the $XT$ plane of the \textbf{Kruskal extension}. The horizons are the dashed lines. The blue wedges are the exterior, the green wedge is the future interior and the past interior is in red. Nothing can escape the green wedge into the blue wedges, because there is no radial null geodesic which would connect the wedges.}\label{kruskal}	
		\end{figure}
	From here on, we will use $r_{s}=2GM=1$.
	Instead of using the coordinates $(t,r,\Omega)$ like in the previous section, we now introduce the Kruskal-Szekeres coordinates, because they are a better choice for near-horizon physics.
	
	First we parametrize the radial null geodesics in the Schwarzschild geometry as
		\begin{equation}
			t=\pm r_{*} + C,
		\end{equation}
	where C is some constant of motion and $r_{*}$ is a new radial coordinate defined as
		\begin{equation} \label{r_*tortoise}
			r_{*}\equiv r+\log (r-1).
		\end{equation}
	also called the \textit{tortoise coordinate}\footnote{The name \textit{"tortoise"} has its origin in the paradoxon of with Achilles and the tortoise.}, because now we have an infinite coordinate range that fits in a finite geodesic distance.
		
		Now, the Kruskal-Szekeres coordinates are defined as
		\begin{equation}
			U\equiv -e^{\frac{r_*-t}{2}}
		\end{equation}
		\begin{equation}
			V\equiv e^{\frac{r_*+t}{2}}.
		\end{equation}
	%\clearpage			%vllt mal Marei fragen, wie man das besser mit den footnotes machen kann
	Their lines of constant U and V are radial null geodesics and these coordinates have the feature, that
		\begin{equation}
			 UV=(1-r)e^r.
		\end{equation}
	This means we have a singularity at $UV=1$ and the horizon is at $U=0$ or $V=0$. The metric looks now like
		\begin{equation}
			\diff s^2=-\frac{2}{r}e^{-r}\left(\diff U \diff V + \diff V \diff U\right)+r^2 \diff \Omega^2_{2}
		\end{equation}
	Because this metric still has an off-diagonal tensor, we define another set of coordinates
		\begin{equation}
		\begin{split}
			U=T-X 	\\	
			V=T+X
		\end{split}
		\end{equation}
	Now the metric looks as follows:
		\begin{equation}\label{SchXT}
			\diff s^2=\frac{4}{r}e^{-r}\left(- \diff T^2+ \diff X^2\right)+r^2 \diff \Omega^2_2
		\end{equation}
	Note that there is now no singularity at $r=1$.
	
	This metric defines a geometry over the full XT plane, which looks like Figure \ref{kruskal}. The right blue wedge is, in the old Schwarzschild coordinates, former $r>1,~-\infty<t<\infty$ HIER FEHLT NOCHWAS. \marginpar{left blue wedge too?}
		
		For $r<1$ we are in the green region for $T>0,~X^2-T^2<0$ and in the red wedge for $T<0$ and still $X^2-T^2<0$. 

	
	%subsection{Penrose diagrams} including supsupsections
	\subsection{Penrose diagrams}
%\FloatBarrier
	We want to know the causal structure of spacetime by asking the question which points can receive signals from which other points. 
For throwing out some irrelevant information, we will know introduce the so called \textit{conformal compactification}.
\begin{figure}[tbp]	
		\begin{center}
		\includegraphics[scale=1]{flatpenrose}
		\end{center}
		\caption{On the left you see the full \textbf{Minkowski space} in pink in the RT plane. We removed the prefactor of \eqref{TRmetric}, because it would diverge at the boundary. On the right we formalize this in to a \textbf{Penrose diagram}.}\label{flatpenrose}
		\end{figure}
%hiermit bin ich noch nicht zufrieden!!!
\parbox{\linewidth}{~}
\hfill
\parbox{0.02\linewidth}{~}
\begin{minipage}[][][c]{0.96\textwidth}
We have two spacetimes with metrics, which are related as $g'_{\mu\nu}(x)=e^{2\omega(x)}g_{\mu\nu}(x)$ with a smooth real function $\omega(x)$. Or in other words, they differ only by multiplication with a positive scalar function on spacetime. The important thing is, those metrics have the same null geodesics.
This goes not without saying, because only timelike/spacelike curves in one metric will be timelike/spacelike curves in the other, but not geodesics.
Two metrics with this kind of relation are called \textit{conformally equivalent}. 
This gives us a way to represent the asymptotic behaviour of spacetimes at large distances.
\end{minipage}
\hfill

\parbox{\linewidth}{~}%hier Abstände nicht verändern!

Now, \textit{conformal compactification} is including infinity as a boundary of spacetime in a manifold by taking a function $\omega(x)$ that diverges while we approach infinity, so that infinity is brought into a finite distance.
\marginpar{sollten metriken damit nicht auch was zu tun haben?}

	\subsubsection{of Minkowski space \checkmark}
		%\clearpage		
		
	We will now show this on an example, the ordinary flat Minkowski space, whose metric in spherical coordinates looks like
		\begin{equation}
			\diff s^2=- \diff t^2+ \diff r^2+r^2 \diff \Omega^2_2
		\end{equation}
	Because the interesting things are happening in the asymptotical behavior at $r \rightarrow \infty$ and $|t| \rightarrow \infty$, we parametrize them with the aid of $\arctan(x)$, so that the boundary is pulled in a finite distance.
		\begin{equation}
		\begin{split}
			T+R\equiv\arctan(t+r) \\
			T-R\equiv\arctan(t-r)
		\end{split}
		\end{equation}
	
	And now the metrics looks as follows:
		\begin{equation} \label{TRmetric}
			\diff s^2=
			\frac{1}{\cos^2(T+R)\cos^2(T-R)}
			\left[- \diff T^2+ \diff R^2+\left(\frac{\sin(2R)}{2}\right)^2 \diff \Omega^2_2 \right].
		\end{equation}	
					
%		\begin{figure}[tbp]	
%		\begin{center}
%		\includegraphics[scale=1]{flatpenrose}
%		\end{center}
%		\caption{On the left you see the full \textbf{Minkowski space} in pink in the RT plane. We removed the prefactor of \eqref{TRmetric}, because it would diverge at the boundary. On the right we formalize this in to a \textbf{Penrose diagram}.}\label{flatpenrose}
%		\end{figure}
	This seems to be quite complicated, but if you have a look at \textbf{Figure \ref{flatpenrose}}, you will see, why we were doing this. The new ranges of our coordinates are $|T \pm R|< \pi/2, R\geq0$, and the spacetime was compactified by including the boundary at $|T \pm R| = \pi/2$. 
	
	The new boundary which are illustrated in \textbf{Figure \ref{flatpenrose}} on the right side, is divided into five parts:	
		\begin{tabbing}
			\hspace{0.1\linewidth} \= \hspace{0.4\linewidth} \= \hspace{0.1\linewidth} \= \hfill \kill
			$i^+$~~: \> future timelike infinity \> $J^+$~~: \> future null infinity \\
			$i^-$~~: \> past timelike infinity \> $J^-$~~: \> past null infinity\\
			\hspace{0.35\linewidth} \= \hspace{0.1\linewidth} \= \hfill \kill	
			~ \> $i^0$~~: \> spatial infinity 
		\end{tabbing}
	So timelike curves come from $i^-$ and go to $i^+$, same for null curves with $J^\mp$, the spatial geodesics are ending at $i^0$. Massless particles are entering/leaving at $i^\mp$ and massive particles at $J^\mp$. The scattering matrix\footnote{also called S-matrix} maps the states on $J^- \cup i^-$ to the states on $J^+ \cup i^+$.	
	
	Out of the diagram, we can see, that the Minkowski space does \textit{not} have \textit{event horizons}.
	%\clearpage
	
	\subsubsection{of de Sitter space \textbf{N}}
	\begin{figure}[tbp]
	 	\begin{center}
			\includegraphics[scale=0.8]{dspenrose}
		\end{center}
	  		\caption{Beschreibung}\label{dspenrose}
	\end{figure}
	Its metric is:
		\begin{equation}
			\diff s^2=- \diff \tau^2+\cosh^2\tau \diff \Omega^2_3
		\end{equation}
		Now we know, that de Sitter space has no infinite spatial boundary $i^0$, nor any light-like infinity $J^\mp$. That means, it is too complicated to find a formulation of de Sitter space in quantum theory and finding a S-matrix may be impossible. 
		
		But it has \textit{event horizons}! That means, observers, who are moving on timelike geodesics at vertical straight lines in the diagram \textbf{Figure \ref{dspenrose}} could be unable to communicate.
	 
	%\FloatBarrier
	\subsubsection{of Schwarzschild geometry \textbf{N}}
	
	For the Penrose diagram of the Schwarzschild geometry, we take the Kruskal Szekeres coordinates of \eqref{SchXT}. That lightens the compactification we need for the Penrose diagram a lot, because the difference between $(T,X,\Omega)$ coordinates and the Minkowski space coordinates $(t,r,\Omega)$ is just their range.\footnote{Minkowski: $-\infty < t < \infty, r\geq 0$; Kruskal: $X^2-T^2 > 1$}
	So the transformation is as it was for Minkowski space:
		\begin{equation}
			\begin{split}
			T'+X'\equiv \arctan(T+X) \\
			T'-X'\equiv \arctan(T-X)
			\end{split}		
		\end{equation}
	But instead we have now the coordinate range of $|X'\pm T'|< \pi/2$ and $|T|< \pi/4$ \marginpar{? warum $|T|$ und nicht $|T'|$}
	If you draw this into a diagramm, it looks like the Kruskal diagram in \textbf{Figure \ref{kruskal}}, but now the spacetime boundaries are shown too.
		\begin{figure}[tbp]  	
	  	\begin{center}
		\includegraphics[scale=1]{schpenrose}
		\end{center}
		\caption{This is the \textbf{Penrose diagramm} for the \textbf{Schwarzschild geometry}. The horizons are marked in dashed lines. As you can see, there are boundaries like the ones of the Minkowski space on both sides.}
		\end{figure}
	\FloatBarrier

		

	%subsection{Classical black hole formation}	
	\subsection{The apparant horizon N}
	\begin{figure}[tbp] 
		\begin{center}
			\includegraphics[scale=1]{collapse}
		\end{center}
			\caption{In both diagrams, the upper boundary is the singularity while the left boundary is the origin of the polar coordinates. The other two boundaries are asymptotic to the ones of the Minkowski space. On the left side we have a collapsing cloud of massive particles shown in orange, which forms the black hole. On the right side we have a black hole forming out of an infalling shell of photons, where we have the Schwarzschild geometry above the orange line, and a piece of Minkowski space below it. The event horizon is illustrated as a dashed line, the apparent horizon is shown with blue dots.}\label{collapse}
	\end{figure}	
	Now, how does that lead us to black holes? First of all, real astrophysical black holes are a result of gravitational collapse, like at the end of a star live. But considering massive particles would mean, that we have to include all interactions between them. So for making it convenient, we instead imagine a black hole arise from a spherically symmetric infalling shell of photons. 
	That also leads to the fact, that there is no obstacle while the black hole is formed, like the Schwarzschild radius.
		
	But now that the horizon extends into the Minkowski space and we have to make a difference between the \textit{actual horizon} and the \textit{apparent horizon}.
	When you are passing the actual horizon, you will not notice it immediately, even though your fate has already been sealed. This leads to some kind of acausal nature of horizons: Their locations depend on events that have not yet happened. 
	
	For defining the apparent horizon, which will help us, to avoid this acasual nature, we first notice, that the Schwarzschild horizon can be detected locally in time, for any sphere of constant $r$ with $r<1$.\footnote{Here, the Schwarzschild radius is set to $2GM=1$}
	Any null geodesics of these spheres which starts out orthogonal, converges towards other null geodesics of this kind. And if we have a compact 2-dim surface, it is called a \textit{closed trapped surface}.
	
	\textbf{An apparent horizon is now a surface which is a boundary of a connected set of closed trapped surfaces.}
	For to describe the real black hole in one diagram, we take a mixture of Minkowski space and Schwarzschild solution into a Penrose diagram as shown in \textbf{Figure \ref{collapse}}. The apparent horizon is illustrated with blue dots. And as you can see, it does form itself in the same moment when the photon shell crosses the event horizon.
					

%\section{What is Quantum field theory?}
\section{What is Quantum field theory? \checkmark}
	Here the Hilbert space is like an infinite tensor product over all points in space, while we have finite degrees of freedom at each point. 
For example, we take a scalar field $\phi(x)$ where there is only a one degree of freedom at each spatial point. 

	Then the free scalar field of mass m's Hamiltionian looks like
		\begin{equation}
			H=\frac{1}{2}\int \diff^3 x \left(\pi(x)^2 + \vec{\nabla}\phi(x) \cdot \vec{\nabla}\phi(x) + m^2\phi(x)^2 \right).
		\end{equation}
	The funktion $\pi(x)$ is the canonical conjugated momentum to $\phi$ which can be put out to tender as $\frac{\partial \mathscr{L}}{\partial \dot{\phi}(x)}$. For beginners, the $\phi$ would be something like the spacial coordinate $x$ in theoretical mechanics and $\pi$ is the companian piece to $p$, the momentum.
	Together they obey
		\begin{align}
			\begin{split}
				[\phi(x),\pi(y)]&=i \delta^3(x-y) \\
				[\phi(x),\phi(y)]&=0 \\
				[\pi(x),\pi(y)]&=0.
			\end{split}
		\end{align}
	This is consistent to the fact that for each field at a each point, we have an individual tensor factor. Note that x and y are four dimensional coordinates living in Minkowski space.
			
	The Hamiltonian results out of the Lorentz-invariant action\footnote{For germanspeakers: Because it is often confused because of television, action in physics means \textit{Wirkung}.}:
		\begin{equation}
			S=-\frac{1}{2} \int \diff^4 x \left(\partial_\mu \phi \partial^\mu \phi + m^2\phi^2 \right)
		\end{equation}
	Where in general action is defined as \cite{MechanikFliesbach}: 
		\begin{equation} 
			S=S[q]= \int\limits_{t_1}^{t_2} \diff t \mathscr{L}(q,\dot q,t)
		\end{equation} 
	In quantum mechanics, we are often interested in describing the ground state wavefunction $|\Omega\rangle$. If we do this for the free massive scalar field, we find that \marginpar{[7]}
		\begin{equation}
			\langle\phi| \Omega\rangle \propto \exp\left[-\frac{1}{2} \int \diff^3 x \diff^3 y \phi(x)\phi(y) K(x,y)\right].
		\end{equation}
	with
		\begin{equation}
			K(x,y)=\int \frac{\diff^3 k}{(2\pi)^3} e^{i\vec{k} \cdot (\vec{x}-\vec{y})} \sqrt{\vec{k}^2+m^2}.
		\end{equation}
	%Here $r \equiv |x-y|$.	
	$K(x,y)$ is a propagator and normally includes at least four coordinates, two for the beginning state and two for the final state. If we want to include time-ordering, we would use the Green's function which would be a Heaviside step function multiplied with this propagator: $G(\textbf{r}_2,t_2;\textbf{r}_1,t_1)=\Theta(t_2-t_1)\cdot K(\textbf{r}_2,t_2;\textbf{r}_1,t_1)$.\cite{QEDbuch}\\
	
	But we now would like to generalize theories with interactions, so we study the vaccum expectation values of products of Heisenberg picture fields which look like $\phi(t,x) \equiv e^{iHt} \phi(x) e^{-iHt}$. In our case of the free massive scalar field we have this solution of equation of motion:
		\begin{equation}  \label{EqOfMotion}
			\phi(t,\vec{x}) = \int \frac{\diff^3 k}{(2\pi)^2} \frac{1}{\sqrt{2\omega_k}} 
			\left[ a_{\vec{k}} e^{i(\vec{k}\cdot\vec{x}-\omega t)} +
			a^{\dagger}_{\vec{k}} e^{-i(\vec{k}\cdot\vec{x}-\omega t)}
			\right]
		\end{equation}
	where we defined $\omega_k \equiv \sqrt{\vec{k}^2+m^2}$. The $a^{\dagger}_{\vec{k}}$ and $a_{\vec{k}}$ should be known out of the basics of quantum mechanics as the creation and annihilation operators\footnote{They are defined: $\hat{a}= \sqrt{\frac{m\omega}{2\hbar}} 
		\left(\hat{q} + \frac{i}{m\omega}\hat{p}
		\right)
		$ and 
		$ \hat{a}^{\dagger}= \sqrt{\frac{m\omega}{2\hbar}} 
		\left(\hat{q} - \frac{i}{m\omega}\hat{p}
		\right)
		$
	with the feature that if you let them operate with a wave function $|n\rangle$ they act like this:
	$\hat{a}|n\rangle = \sqrt{n}|n-1\rangle$ and $\hat{a}^{\dagger}|n\rangle= \sqrt{n+1}|n+1\rangle$.
	And $a^{\dagger}$ is the adjoint of $a$ which means: $\langle \psi | a | \phi \rangle^*= \langle \phi | a^{\dagger} | \psi \rangle$ just for reminding.
%end footnote
	}, and they obey
		\begin{align*}
			[a_{\vec{k}},a^{\dagger}_{\vec{k'}}]&= (2\pi)^3 \delta^3 (\vec{k}-\vec{k'})\\
			[a_{\vec{k}},a_{\vec{k'}}]&=0 \\
			[a^{\dagger}_{\vec{k}},a^{\dagger}_{\vec{k'}}]&=0 \\
			[H,a_{\vec{k}}]&=-\omega_k a_{\vec{k}}.
		\end{align*}
	Now we are searching for a more abstract form of \eqref{EqOfMotion} where $a_n$ and $a^{\dagger}_n$ have the standard algebra, in the following way:
		\begin{equation} \label{wave_fct}
			\phi = \sum_n \left(f_n a_n + f^*_n a^{\dagger}_n \right)
		\end{equation}
	Here we use a basis of solutions $f_n(x)$ of
		\begin{equation} \label{Klein-Gordon-eq}
			\left( \partial_{\mu}\partial^{\mu} - m^2 \right) f(t,\vec{x})=0
		\end{equation}			
	that have a time dependence of the form $e^{-i\omega t}$ with $\omega>0$. We also define the same $f_n$ in a way, that they are orthonormal in the Klein-Gordon\footnote{Perhaps you already noticed, that \eqref{Klein-Gordon-eq} is the Klein-Gordon Equation which is nothing else than the relativistic Schrödinger Equation. It is just valid for spin 0 particles.} norm, so that
		\begin{equation}
			(f_1,f_2)_{KG} \equiv
			i \int d^3x \left( f^*_1 \dot{f}_2 - \dot{f}^*_1 f_2
			\right).
		\end{equation}
	Now we can have a look at the expectation values (perhaps you remember, that this is the interesting part), which are called \textit{correlation functions}. With just a one-point function, it has to vanish
		\begin{equation}
			\langle \Omega| \phi(t,\vec{x}) |\Omega \rangle = 0.
		\end{equation}
	because of the translation invariance in vacuum and how $a_n$ and $a^{\dagger}_n$ act on the vacuum. In other words, it is forbidden, that one particle just arise from the vacuum but never vanishes oder that one particle had always existed end suddenly disappears. This is, why Quantum field theory is called a multiparticle theory.
	
	For a two-point function with equal times it looks like this:
		\begin{equation}
			\langle \Omega| \phi(0,\vec{x})\phi(0,\vec{y}) |\Omega\rangle=
			\frac{1}{4\pi^2}\frac{m}{|\vec{x}-\vec{y}|}K_1(m|\vec{x}-\vec{y}|).
		\end{equation}
	This function scales with 
		\begin{itemize}
			\item[•] $\frac{1}{|\vec{x}-\vec{y}|}$ for $|\vec{x}-\vec{y}| \ll m^{-1}$
			\item[•] $e^{-m|\vec{x}-\vec{y}|}$ for $|\vec{x}-\vec{y}| \gg m^{-1}$
		\end{itemize}
	but is "gapless" for $m=0$, because the $m^{-1}$, which is also called \textit{correlation length}, is infinite, so exicted states' energies can be arbitrarily close to the ground state energy. A massless scalar field is also invariant under the \textit{conformal group}. This means here we can scale the spacetime like $x'^\mu = \lambda x^\mu$, so it's scale-invariant. The quantum field theory with this larger symmetry group is called \textit{conformal field theory} or \textit{CFT}.
	
	$K_1(m|\vec{x}-\vec{y}|)$ is some hyperbolic Besselfunction which is in general a solution of differential equation like the Klein-Gordon equation in hyperbolic coordinates. It contains an exponential function, which is, why we have the second point in the listing above. But I will not do further explanations about this function because it is not necessary here. \\
	
	Now that we contemplate the two-point function without the time, we add the time difference to our calculations. But we need to order the function in time so we use the time-ordering operator $T$, so that the two-point function now is:
		\begin{equation}
			\langle \Omega| T \phi(t,\vec{x})\phi(t',\vec{y}) |\Omega\rangle=
			\frac{1}{4\pi^2} \frac{m}{\sqrt{|\vec{x}-\vec{y}|^2-(t-t')^2+i\epsilon}}
			K_1 \left( m\sqrt{|\vec{x}-\vec{y}|^2-(t-t')^2+i\epsilon}\right)
		\end{equation} 
	This is also called a \textit{transition amplitude}. T takes care that time increases. The factor $\epsilon$ should go to zero and reminds us, that there could be other ways of complex integration because of the square root not leading to the timelike-seperated points.	

%\section{What is Entanglement and what is it good for?}
\section{What is Entanglement and what is it good for? \textbf{N}}
	In relativistic QFT, the ground state has correlations between field operators at spatially separated points. Here we can use \textit{entanglement} as an explanation.
	
	But at first, let's start from the beginning:
	\\
	We have $\rho$ which is called \textit{density matrix} and is a quantum state on Hilbert space $\Hil$. Quantum states are illustrated in operators, here: $\rho$ is a non-negative hermitian one of trace 1. If it can be written in the form\footnote{Here $|\psi\rangle$ is some element of $\Hil$ with norm 1.}
		\begin{equation}
			\rho=|\psi\rangle \langle\psi|,
		\end{equation}
	 the quantum state is called \textit{pure}. If a state is not pure, it is \textit{mixed}.
	 
	 While doing an experiment, we will a measure a outcome $i$, which is always related to a projection operator $\Pi_i$ with a probability of measuring $i$, that looks like: 
		 \begin{equation}
	 		P(i)=\mathrm{tr}(\rho\Pi_i).
	 	\end{equation}
	 \begin{figure}[tbp]
	 	\begin{center}
	 		\includegraphics[scale=1]{entangledcorr}
	 		\caption{The boundary between these two regions is called the \textit{entangling surface}.}
	 	\end{center}
	 \end{figure}
	 For to find out, whether a given state $\rho$ is pure or mixed, we define a function $S$ for convenience:
		\begin{equation}
			S(\rho)\equiv -\mathrm{tr}(\rho \log \rho)
		\end{equation}	 	
	And $S(\rho)$ is called the \textit{Von Neumann Entropy} or \textit{information entropy}. 
	Its proberties are:
	\FloatBarrier
	\begin{itemize}
		\item[•] for any unitary operator $U$: $S(U^\dagger \rho U)=S(\rho)$
		\item[•] $S(\rho)\geq 0$, with equality if and only if $\rho$ is pure. 
		\item[•] for $d$ is the dimension of $\Hil$: $S(\rho)\leq \log d$, with equality if and only if $\rho$ is maximally mixed.
		\item[•] The entropy of the average over a set of states is at least equal to the average of all their individual entropies. This is also called \textit{concavity} and is definde as:
		\begin{equation}
			S \left(\sum_i \lambda_i \rho_i \right) \geq \sum_i \lambda_i S(\rho_i),
		\end{equation}
			while $\lambda_i$ is any set of non-negative numbers with $\sum_i \lambda_i =1$.
	\end{itemize}
	\FloatBarrier
	Now, let's have a look at an entangled state written in the two-qubit state:
		\begin{equation}
			|\Psi\rangle = \frac{1}{\sqrt{2}} \left(|00\rangle + |11\rangle \right)
		\end{equation}
	Here, the full state is pure, but the reduced state on either qubit ($|00\rangle$ or $|11\rangle$) is mixed. 
%	Bipartite systems are the ones whose Hilbert space can be written as a tensor product\footnote{Watch out! The tensor product $\otimes$ is not the direct sum $\oplus$.}:
%	 	\begin{equation}
%	 		\Hil=\Hil_A \otimes \Hil_B
%	 	\end{equation}
%	 In quantum mechanics they describe the composition of two independent physical systems and can be \textit{entangled}, which means, that the reduced density matrices $\rho_A$ and $\rho_B$ can be mixed even if the joint state $\rho_{AB}$ is pure.
%	 

\FloatBarrier 

%\section{Rindler}
\section{Rindler decomposition}
\FloatBarrier
	bla
		\begin{figure}[tbp]
			\begin{center}
				\includegraphics[scale=1]{boost}
				\caption{The \textbf{Rindler decomposition} of \textbf{Minkowski space}. The blue wedges are the Rindler wedges, the red one is the past wedge and the green one is the future wedge. The straight lines in black are slices of the Rindler time. The \textit{red lines} are the action of the \textit{boost operator} $K_x$.}\label{Rindler}
			\end{center}
		\end{figure}
	Now that we know  what entanglement is, we would also like to know, with whom is who entangled. In \textbf{Figure \ref{Rindler}} you can see a methode which will help us, to reach that goal, the Rindler decomposition of Minkowski space.
	
	Therefore we split the Hilbert space into a factor $\Hil_L$ that acts on the fields $x<0$ and $\Hil_R$ for $x>0$. And each factor has its own basis of states with which we can decompose the vacuum. \marginpar{was meint hier "aufteilen"}
	
	We now introduce the \textit{Lorentz boost\footnote{A Lorentz boost is a rotation-free Lorentz transformation, which is a Gallilei-transformation in relativistic.[ARTfließbach p.7]} operator} $K_x$, which mixes x and t, but leaves y and z like they are, because otherwise we could not map it on paper like in \textbf{Figure \ref{Rindler}}. This operator exist in any relativistic QFT and looks in the free massive theory like this:
	\begin{equation}
		K_x = \frac{1}{2} \int \diff^3 x 
		\left[ x (\dot{\phi} + \vec{\nabla} \phi \cdot \vec{\nabla} \phi + m^2 \phi^2) + t\dot{\phi} \partial_x \phi
		\right].
	\end{equation}
	It is not explicitly time-dependant, because the time-dependance of fields in the Heisenberg picture\footnote{Here the operators are following an equation of motion. [see Schwabl QM1 p.176]} is canceling the time-dependance of $K_x$ out. 
	
	 
	\subsection{What is Rindler space?}
	
	
	\subsection{What are Rindlers wedges?}

	\subsection{Entanglement in the Rindler decomposition \checkmark}
	What happens, if we want to cross the $x=0$ surface? For finding that out, we put the system in a mixed state with
		\begin{equation} \label{mixed_state_firewall}
			\rho=\rho_L \otimes \rho_R
		\end{equation}
	instead of having a ground state $|\Omega\rangle$. Here $\rho_L$ and $\rho_R$ are the thermal density matrices, which we get if we trace out the respectively other one in the vacuum $|\Omega\rangle$.
	If the fields are completely discontinous like in \eqref{mixed_state_firewall} , the gradient term of the Hamiltonian will diverge at $x=0$. If you are an observer in the left or right Rindlers wedge it seems, that you just have vacuum state, but the energy is infinit. 
	
	Its typical field fluctuation is given by $\frac{1}{\epsilon}$ where $\epsilon$ is a short-distance length cutoff. \marginpar{was ist damit genau gemeint?} So it is valid that
		\begin{equation}
			\partial_x \phi|_{x=0} \propto \frac{1}{\epsilon^2}.
		\end{equation}
	Which means, that the gradient term in the Hamiltonian contributes
		\begin{equation}
			\diff x \int \diff^2 y (\partial_x\phi)^2 \propto 
			\epsilon \int \diff^2 y \frac{1}{\epsilon^4} 
			= \frac{A}{\epsilon^3}
		\end{equation}	
	The smaller $\epsilon$ is the bigger becomes the energy and $\epsilon$ is even to the third power.
	 			
	This is called a \textit{firewall}: A huge concentration of energy at $x=0$, that annihilates anybody who tries to jump through the Rindler horizon into the future wedge.
	
	For example the product states $|00\rangle$ and $|11\rangle$ of the states $\frac{1}{\sqrt{2}}
	(|00\rangle \pm |11\rangle)$ which shall both have smooth horizons, should have them too, because of linearity of quantum mechanics. But as we just saw, no product state possibly can have a smooth horizon in QFT. So, for going smoothly through the Rindlers horizon we need not only any entanglement but it must have the \textit{right entanglement} too.
	
	%subsection{Eigenstate and Euklidean path integral in general}
	\subsection{Eigenstate and Euklidean path integral in general \checkmark}
	We now introduce the Euclidean path integral because we want to find the eigenstates of the left and right Rindler wedges. 
	
	First of all we have a ground state $\ket{\Omega}$ of $H$ and a vacuum state $\ket{0}$, so we can write for a very long time $T$
	\begin{align*}
		e^{-iHT}\ket{0} &= \sum_n e^{-iE_n T} \ket{n}\braket{n|0} \\
		&= e^{-iE_0 T} \ket{\Omega} \braket{\Omega|0} 
		+ \sum_{n\neq 0} e^{-iE_n T} \ket{n} \braket{n|0}
	\end{align*}
	Now we enclose and get the ground state
	\begin{align*}
		\ket{\Omega} &= \lim\limits_{T \rightarrow \infty} \frac{e^{iE_0 T}}{\braket{\Omega|0}} \,e^{-iHT} \ket{0}
	\end{align*}
	We can define $E_0$ with $H_0\ket{0}=\ket{0}$ so
	\begin{align*}
		\ket{\Omega} &=  \frac{1}{\braket{\Omega|0}} \lim\limits_{T \rightarrow \infty} e^{-iHT} \ket{0}
	\end{align*}
	(see p.86 in \cite{PaS}) But this equation does still tell us nothing about the entanglement. So let's continue: \\
	Let a time-independent field $\phi$ %, which is a field configuration at $t=0$,
	act on this ground state:
	\begin{align*} \label{groundstate_phi}
		\braket{\phi|\Omega} &= \frac{1}{\braket{\Omega|0}} \lim\limits_{T\rightarrow \infty} \braket{\phi|e^{-iHT}|0} 
	\end{align*}
	Now use the Euclidean path integral formalism, rotate $t$ about 90$^\circ$ into the complex plane: $t \rightarrow -i t_E$ and choose the early boundary condition $\phi=0$, so that
	\begin{align}
		\braket{\phi|\Omega} \propto \int_{\hat{\phi}(t_E = -\infty)= 0}^{\hat{\phi}(t_E = 0)=\phi} D\hat{\phi}e^{-I_E}
	\end{align}
	with the Euclidean action for a free massive scalar field
	\begin{align}
		I_E[\hat{\phi}] = \frac{1}{2} \int \diff^3x \diff t_E \left[
			(\partial_{t_E}\hat{\phi})^2 
			+ (\vec{\nabla}\hat{\phi})^2
			+ m^2\hat{\phi}^2
		\right]
	\end{align}
	Note that $\hat{\phi}$ compared to $\phi$ is time-dependent.
	
	%subsection{The ground states of the Rindler wedges}
	\subsection{The ground states of the Rindler wedges \checkmark (bis auf Fragen)} \label{groundstate_rindler_wedges}
	\begin{figure}[tbp]
		\begin{center}
			\includegraphics[scale=1]{eucpath}
			\caption{This is the Euclidean path integral representation changing \eqref{groundstate_phi} into a calculable integral, after we choose to integrate over an angle $\pi$. Note that $\varphi_R$ and $\varphi_L$ are the $\phi$s in the text.}\label{eucpath}
		\end{center}
	\end{figure}
	But instead of integrating from $t_E= -\infty$ to 0, we choose the range of $\pi$ like in \textbf{Figure~\ref{eucpath}}. In addition our Hilbert space is separated in to $\Hil_L$ and $\Hil_R$ with the fields $\phi_L$ and $\phi_R$ so we write 
	\begin{align} \label{pathint}
		\braket{\phi_L \phi_R | \Omega} \propto \braket{\phi_R| e^{-\pi K_R} \Theta| \phi_L}_L
	\end{align}
	Here the $K_R$ is the operator $K_x$ but in the right Rindler wedge while the operator $\Theta$ which is antiunitary\footnote{unitary would mean for a linear operator $A$: $A^\dagger A = \mathds{1}$ so $\braket{Ax|Ay}=\braket{x|y}$, antiunitary would be a antilinear operator $A$ ($\braket{x|A^\dagger}=\braket{y|Ax}$) which also fulfills: $\braket{Ax|Ay}=\braket{y|x}$. },also called CPT and exists in all quantum field theories. It acts on a scalar field in the Heisenberg picture like $\Theta^\dagger \phi(t,x,y,z)\Theta = \phi^\dagger(-t,-x,y,z)$ which gives a map between the to Hilbert spaces. This is important in \eqref{pathint} because \marginpar{warum e hoch -pi, was ist das i in der summe, was ist omega i und v.a. warum steht manchmal ein L oder ein R außerhalb der Kets}
		\begin{enumerate}
			\item $\braket{\phi_L \phi_R| \Omega}$ can now be described just with a matrix in $\Hil_R$
			\item the $\phi_L$ is playing the role of a final state, the $\phi_R$ the role of the initial state in \textbf{Figure \ref{eucpath}}.
		\end{enumerate}
	Now we can evaluate \eqref{eucpath}:
	\begin{align}
		\braket{\phi|\Omega} &\propto \sum_i e^{-\pi \omega_i} \braket{i | \Theta | \phi_L} \braket{\phi_R|i}_R \nonumber\\
		&\propto \sum_i e^{-\pi \omega_i} \braket{\phi_L| i^*}_L\braket{\phi_R|i}_R
	\end{align} %nachrechnen
	We inserted a complete set of  $K_R$ eigenstates, used that $\Theta$ is antiunitary and defined: $\ket{i^*}_L = \Theta^\dagger \ket{i}_R$, so now the ground state is:
	\begin{align} \label{groundstate}
		\ket{\Omega} = \frac{1}{\sqrt{Z}} \sum_i  e^{-\pi \omega_i} \ket{i^*}_L \ket{i}_R 
	\end{align}
	In this connection Z is the partition function, a constant which is given through the normalization condition\footnote{The sum over all probabilities is equal to one. see p.11 \cite{Brenig}}. 
	
	Or we compute the density matrix of the right Rindler wedge (because see above: we just need the matrix of $\Hil_R$)
	\begin{align}
		\rho_R= \frac{1}{Z} \sum_i e^{-2\pi\omega_i} \ket{i}_R \bra{i}
	\end{align}
	which is the thermal density matrix with temperature $T=\frac{1}{2\pi}$. 
	\marginpar{warum T dimensionslos?}

	
	%subsection{In free massive scalar theory and Unruh temperature}
	\subsection{In free massive scalar theory}	
	Let's search solutions for the free massive scalar theory in Rindler decomposition to have an example. First, we introduce hyperboloidal coordinates for the right and left wedge in \textbf{Figure \ref{Rindler}} omitting a length scale $\ell$ because of simplicity (see next subsection \ref{Unruh}).
	\begin{align} \label{rindler_coordinates}
		\begin{split}
		x &= e^{\xi_R} \cosh \tau_R = -e^{-\xi_L} \cosh \tau_L \\
		t &= e^{\xi_R} \sinh \tau_R = e^{-\xi_L} \sinh \tau_L
		\end{split}
	\end{align}
	The coordinate ranges are: $-\infty < \xi_{L,R} < \infty,~ -\infty < \tau_{L,R} < \infty$.
	And the translation of $\tau_R$ forwards and $\tau_L$ backwards in time is the evolution of the boost operator $K_x$, while the $\xi_{L,R}$ coordinates form hyperboloidal orbits of $K_x$. They only cover the blue wedges of \ref{Rindler} but not the future or past Rindler wedges. There are so called Rindler horizons at $\xi_R=-\infty$ and $\xi_L=\infty$ but they are again just there because of choice of coordinates and not actual event horizons. 
	We get the corresponding metric by plugging in \eqref{rindler_coordinates} into $\diff s^2 = -\diff t^2 + \diff x^2 + \diff \vec{y}^2$:
	\begin{align}
		\diff s^2 = e^{2\xi_R} (-\diff \tau_R^2 + \diff \xi_R^2) + \diff \vec{y}^2 
		= e^{-2\xi_L} (-\diff \tau_L^2 + \diff \xi_L^2) + \diff \vec{y}^2
 	\end{align}
 	The solutions of the massive wave equation should now be of the form:
	\begin{align}
		\begin{split}
		f_{R\omega k} &= e^{-i\omega \tau_R} e^{i\vec{k}\dot \vec{y}} \psi_{Rk\omega}(\xi_R) \\
		f_{L\omega k} &= e^{-i\omega \tau_L} e^{i\vec{k}\dot \vec{y}} \psi_{Lk\omega}(\xi_L)
		\end{split}
	\end{align}
	Note that $\omega > 0$ and the $\psi$s obey the equations
	\begin{align}
	\begin{split}
		\left[
			-\partial^2_{\xi_R} + (m^2 + \vec{k}^2) e^{2\xi_R} - \omega^2
		\right] \psi_{Rk\omega} &= 0 \\
		\left[
			-\partial^2_{\xi_L} + (m^2 + \vec{k}^2) e^{-2\xi_L} - \omega^2
		\right] \psi_{Lk\omega} &= 0
	\end{split}
	\end{align} 
\subsection{The Unruh temperature N} \label{Unruh}
	
	%subsection{Entanglement in the Rindler decomposition}
	\subsection{Entanglement in the Rindler decomposition \checkmark}
	What happens, if we want to cross the $x=0$ surface? In order to find that out, we put the system in a mixed state with
		\begin{equation} \label{mixed_state_firewall}
			\rho=\rho_L \otimes \rho_R
		\end{equation}
	instead of having a ground state $|\Omega\rangle$. Here $\rho_L$ and $\rho_R$ are the thermal density matrices, which we get if we trace out the respectively other one in the vacuum $|\Omega\rangle$.
	If the fields are completely discontinuous like in \eqref{mixed_state_firewall} , the gradient term of the Hamiltonian will diverge at $x=0$. If you are an observer in the left or right Rindlers wedge it seems, that you just have vacuum state, but the energy is infinite. 
	
	Its typical field fluctuation is given by $\frac{1}{\epsilon}$ where $\epsilon$ is a short-distance length cutoff. \marginpar{was ist damit genau gemeint?} So it is valid that
		\begin{equation}
			\partial_x \phi|_{x=0} \propto \frac{1}{\epsilon^2}.
		\end{equation}
	Which means, that the gradient term in the Hamiltonian contributes
		\begin{equation}
			\diff x \int \diff^2 y (\partial_x\phi)^2 \propto 
			\epsilon \int \diff^2 y \frac{1}{\epsilon^4} 
			= \frac{A}{\epsilon^3}
		\end{equation}	
	The smaller $\epsilon$ is the bigger becomes the energy and $\epsilon$ is even to the third power.
	 			
	This is called a \textit{firewall}: A huge concentration of energy at $x=0$, that annihilates anybody who tries to jump through the Rindler horizon into the future wedge.
	
	For example the product states $|00\rangle$ and $|11\rangle$ of the states $\frac{1}{\sqrt{2}}
	(|00\rangle \pm |11\rangle)$ which shall both have smooth horizons, should have them too, because of linearity of quantum mechanics. But as we just saw, no product state possibly can have a smooth horizon in QFT. So, for going smoothly through the Rindlers horizon we need not only any entanglement but it must have the \textit{right entanglement} too.

%\section{Contemplation of a fixed black hole}
\section{Contemplation of a fixed black hole}

\subsection{Approximation from Schwarzschild to Rindler \checkmark} \label{approxChap}
Upto now we still considered the quantum field theory and classical black holes apart from each other. Combining them would require a theory of quantum gravity on which i will not go deeper in this thesis. But we can start with a much simpler problem that would be the quantum field theory in a fixed black hole backround. Physically this means, that we're sending $G$ to zero and the mass of the black hole $M$ to infinity so that the Schwarzschild radius $r_s = 2GM$ ist fixed. This can be assured by sending $\frac{M}{m_p}$ to infinity, where $m_p =\frac{1}{\sqrt{8\pi G}}$ is the Planck mass. 

This approximation is justified, which we can see by the example of a black hole with the mass of our sun:
	\begin{align}
		\frac{m_p}{M_{solar}} \approx 2.4 \cdot 10^{-39}
	\end{align}	 
In this type of scale, the Schwarzschild radius is in order of kilometers (sun: $r_s=3 \unit{km}$) where we could imagine doing experiments. 

Now we must descide, which geometry we use. Either the two-sided Schwarzschild geometry or the one-sided collapse geometry. The disadvantage of the latter is, that it is only Schwarzschild after the infalling matter has gone in. So for avoiding the infalling shell problem, we use the former. 
The Schwarzschild geometry is approximatly the region of Minkowski space which is near the Rindler horizon in the Rindler decomposition, if $r \approx 1$ and the angular arrangement is not quite too big. We now sketch this for the right exterior ($r>1$) by using the tortoise coordinate from \eqref{r_*tortoise}:
	\begin{align*}
		r_*=r + log(r-1)
	\end{align*}
The Schwarzschild metric then is
	\begin{align}
		\diff s^2=\frac{r-1}{r}\left(-\diff t^2 + \diff r^2_*\right) + r^2 \diff \Omega^2_2
	\end{align}
Now let's say $y_1=\theta, y_2=\varphi$ where the two angulars are orthogonal coordinates on the sphere. Near the horizon $r\approx1$ the metric then looks like 
	\begin{align}
		\diff s^2 \approx e^{r_* - 1} \left(-\diff t^2 + \diff r^2_*\right) + \diff \vec{y}^2
	\end{align}
This quite resembles the right Rindler wedge metric. It even becomes equal, if we insert
	\begin{align} \label{again_rindler_coord}
		\begin{split}
			r_*&= 2\xi_R + 1 - \log 4 \\
			t &= 2\tau_R
		\end{split}
	\end{align}
which leads to
	\begin{align*}
		&\Rightarrow &
		\diff t^2 &= 4 \diff \tau_R^2 & \diff r_*^2 &= 4 \diff \xi_R^2 &e^{r_* -1} &= e^{2\xi_R + 1 - 1 - \log 4} & \diff \vec{y}^2 &= \diff \vec{y}^2
	\end{align*}
and finaly like in \eqref{Rindler_metric}
	\begin{align*}
		\diff s^2 = e^{2\xi_R} (-\diff \tau_R^2 + \diff \xi_R^2) + \diff \vec{y}^2 
	\end{align*}
How this looks like in Penrose diagrams if we do the same thing for the other three wedges too, can be seen in \textbf{Figure \ref{approximation}}.
\begin{figure}[tb]
	\begin{center}
		\includegraphics[scale=1]{rindsch} \label{approximation}
		\caption{At the left, one can see the Rindler Penrose, at the right there is the Schwarzschild Penrose. The two grey regions in the middle are the regions near the horizon. As you can see, they approximate each other well. Note that a $\mathds{R}^2$ is suppressed at each point instead of $\mathds{S}^2$ which is why the left Penrose doesn't look like the Minkowski Penrose in \textbf{Figure \ref{flatpenrose}}.}\label{approximation}
	\end{center}
\end{figure}
The outcome of this is, that for every initial state in a Schwarzschild geometry the left and right exteriors must be thermally entangled, assumed that the Schwarzschild looks like Minkowski space near them. 
The relation \eqref{again_rindler_coord} between Schwarzschild and Rindler time leads to 
	\begin{align}
		T_{\text{Hawking}} = \frac{T_{\text{Unruh}}}{2} = \frac{1}{4\pi}
	\end{align}
This alliance between temperature and time can be made, because the time one is exposed to the radiation is proportional to the temperature. 

With all constants the Hawking temperature is
	\begin{align}
		T_{\text{Hawking}} = \frac{1}{4\pi r_s} = \frac{\hbar c^3}{8 \pi k_B GM}
	\end{align}
So, the bigger the temperature is, the bigger is the size of the black hole. For a solarmass one for example, the temperature is about $6 \cdot 10^{-8} \,\unit{K}$. 
	
	%\\subsection{Schwarzschild modes}
	\section{Schwarzschild modes \textbf{N}}
	\begin{figure} [tbp]
		\begin{center}
			\includegraphics[scale=0.5]{plots_of_V}
			\caption{These are the plots of $V(r_*)$ for $l=\{0,1,2,3$\}.} \label{plots_of_V}
		\end{center}
	\end{figure} \todo{eigener Plot eingefügen}

	We now have a look at free fields in the Schwarzschild geometry. For the beginning we need to find a family of modes $f$ that solve the free scalar equation of motion
		\begin{equation} \label{Klein_Gordon_curved}
			\frac{1}{\sqrt{-g}} \partial_\mu (\sqrt{-g} g^{\mu \nu} \partial_\nu \phi)
			= m^2 \phi
		\end{equation}
	which is the \textit{Klein-Gordons equation for curved space} and where $g_{\mu\nu}$ is the Schwarzschild metric and $g$ is its determinant. Its solutions are also modes in \eqref{wave_fct}, were we can study its properties in an appropriate quantum state such as the Hartle-Hawking state\footnote{what is that}.\todo{was ist das?}
	
	We now focus on the right exterior of the Schwarzschild geometry, where we use the coordinates $(t,r, \Omega)$. The solutions are having the form
		\begin{equation}
			f_{\omega l m} = 
			\frac{1}{r} Y_{l m}(\Omega) e^{-i \omega t} \psi_{\omega l}(r)
		\end{equation}
	Let's put these into the equation \eqref{Klein_Gordon_curved} above and use the tortoise coordinates from \eqref{r_*tortoise} to reform it into a Schrödinger equation:
		\begin{equation}
			- \frac{d^2}{dr^2_*} \Psi_{\omega l} 
			+ V(r) \Psi_{\omega l} 
			= \omega^2 \Psi_{\omega l}
		\end{equation}
	with the effective Potential of
		\begin{equation}
			V(r)=
			\frac{r-1}{r^3} \left( m^2r^2 + l(l+1) + \frac{1}{r}
			\right)
		\end{equation}
	Let's have a look at the mass m: For simlicity, we consider the Compton wavelength $\frac{1}{m}$ can be\footnote{ The \textit{Schwarzschild radius $r_s$ is still 1}, but I sometimes write $r_s$ to make some things more obvious.} 
		\begin{enumerate}[(i)]
			\item $\frac{1}{m} \ll r_s$	~~ which is the \textit{massive case} and \label{massive}
			\item $\frac{1}{m} \gg r_s$	~~ which is the \textit{massless case}. \label{massless}
		\end{enumerate}
	For to explain, why case \eqref{massless} is more interesting for us, we first need to have a look at case \eqref{massive}.
	
	 Here the potential goes to $m^2$ for $r \gg 1$ which means that massive modes will only propagate till near infinity if $\omega \geq m$. Because we assumed, that $m \gg 1$, any modes with an energy $\omega$ of order of the Schwarzschild radius \textit{will stay near the horizon}. In addition the temperature of black holes\footnote{The Hawling-temperatur is defined as $T_{Hawking}= 
	 \frac{\hbar c^3}{4 \pi k_B r_s}$.} is of order $\frac{1}{r_s}$. This means, that $\omega \approx 1$ would be the most interesting energy. \marginpar{warum massive case dann uninteressant?}
	
	So from now on we remain within the case of $m^2=0$.
	Here the asymptotic behavior of the potential is
		\begin{equation}
			V\approx
			\begin{cases}
				\frac{l(l+1)}{r^2_*} &r_* \rightarrow \infty \\
				(l^2 + l + 1) e^{r_*-1} &r_* \rightarrow - \infty
			\end{cases}
		\end{equation}
		
	\begin{figure} [tbp]
		\begin{center}
			\includegraphics[scale=1.6]{schscat}
			\caption{Bla} \label{plots_of_V}
		\end{center}
	\end{figure}

%\section{Information problem}
\section{Information problem}
	
	%\subsection{Black hole radiation according to Hawking}
	\subsection{Black hole radiation according to Hawking \checkmark} 
	We now have a closer look to a one-sided black hole which was created by a collapse. As discribed in the caption of \textbf{Figure \ref{collapse}}, just the section above the shell is Schwarzschild geometry, below we have Minkowski geometry. This means, that the solutions we found in the section before are just usefull for above the shell. The solutions below the shell come from the Minkowski wave equation and are plaine wave kind. The first must get matched with the second, otherwise we won't have a solution for the whole space-time.
	
	The difficulty is, that there even is no global time-translation symmetry. While modes which are in Schwarzschild time are having positive frequency, the modes in Minkowski propagate with positive and negative frequency. So the modes coming from above the shell and are reflected stay positive because of energy conserving, while the modes who travel through the shell are getting negative frequency. 
	
	As early time definition we set the Minkowski modes to nonexited ones. The consequence is, that the modes in early Schwarzschild time are exited. 
	To figure out how much these modes are exited, we need to find a relation between the creation and annihilation operators of Minkowski modes and Schwarzschild modes. This is very complicated and you can read about in Hawking's orignial paper \cite{Hawking}. In the end we get the floating energy in a band of late-time outgoing modes $f_{\omega \ell m}$, which are shown in \textbf{Figure \ref{scattering}} as the orange arrow and also in \textbf{Figure \ref{collapse2}}, with width $\diff \omega$: 
	\begin{align} \label{energy_flux}
		\frac{\diff E}{\diff t} = \frac{\omega \diff \omega}{2\pi}
		\frac{P_{abs}(\omega,\ell)}{e^{\beta \omega}-1}
	\end{align} 
	The $\beta$ is equal to the inverse of $T_{\text{Hawking}}$ while $P_{abs}(\omega,\ell)$ is the probablility for a mode with frequency $\omega$ and angular momentum $\ell$ coming from the right in \textbf{Figure \ref{scattering}} to transmit through the barrier.
	Without this factor, \eqref{energy_flux} would be like the standard formular for the radiation of a black body with temperature $T = \beta^{-1}$ in vacuum. 
	\begin{figure}
		\begin{center}
			\includegraphics[scale=1]{collapse2}
			\caption{This is the geometry for a black hole made from collapse. Other than in \textbf{Figure \ref{collapse}} the collapsing shell is here shown in blue. The orange dashed line is the late time outgoing wave packet but in backwards evolution. This means, if we would go forwards in time, the arrow would be in the other direction. in seperates because some of it gets reflected off the shell and some goes through the shell.} \label{collapse2}
		\end{center}
	\end{figure}
	
	Please don't get confused by the directions we are talking about. The probability is for the wave packet evoluting backwards in time like in \textbf{Figure \ref{collapse2}} shown, but we are actually talking about the modes escaping the black hole by traveling forwards in time.
	 
	The barrier's height is proportional to $\ell^2$, so the most emitted modes will be the ones with the lowest angular momentum. This means the quantum states in the field near the black hole formed by the collapse, there are nearly stationary modes which are thermally exited and mostly the ones with low $\ell$ can carry energy to infinity after tunneling through the barrier. 
	
	Let's consider shortly the spins. In a massless field the particles with higher spin will carry less energy from near the horizon into the infinity. This is the case because the modes of particles with spin only radiates with $\ell >0$. And the higher the spin, the bigger the angular momentum of the modes will get. Thus the potential barrier will also grow. In the end this means, that most of the energy will be carried out to infinity by the massless particles with the lowest spin. In our universe these would be the photons while the after them the gravitons are carring a smaller, but still mentionable part of energy.\\
	
	Hawking used a picture in popular science to explane how the radiation was generated. He explained that due to quantum fluctuation, particles and antiparticles jup to excistence and disappear, but near the horizon, one is drifting into the black hole while the other travels into infinity and can be measured. There are several problems with this picture. For example, the particle, that falls into the black hole must have negativ energy, means negative mass. But as far as we now, there is no such thing as a anticharge to mass. Next Problem is, that the wavelengths of such 'particles' would depend on the mass of the black hole, so they can't be located near the horizon. Unfortunately this picture is the most popular explanation of Hawking radiation.
	
	
	%\subsection{Evaporation}
	\subsection{Evaporation N}
	Until now the black hole could radiate an infinit amount of energy. This is of course not physically correct. We will now resore the dynamical gravity so that the mass becomes time independent. This means, it is no longer reasonable to set the Schwarzschild radius to one. From now on $r_s= 2GM$.	
	
	So the old definition of the black hole's energy was that is generated translations of $t$ in the Schwarzschild geometry. But in the general relativity it doesn't work any more to attach this definition to the coordinates.
	
	In an asymtotical flat space and using Hamilton formalism while the general relativity is inlcuded, the energy is an integration over the boundary at the two sphere at $r\rightarrow \infty$. In the Penrose diagram, this would be the $i^0$ boundary. It is also called the ADM energy\footnote{cherished to Arnowitt, Deser and Misner} and is not only conserved but for a black hole with mass $M$, the ADM energy is also $M$, if it was formed form a collapsing shell. 
	
	Now we might be able to calculate how long a black hole lives.
	First we need the total energy disposal:
	\begin{align}
		\frac{\diff E}{\diff t} =
		\sum_{\ell, m} \int_{0}^{\infty} \frac{\diff \omega}{2 \pi}
		\frac{\omega P_{abs}(\omega,\ell)}{e^{\beta \omega} - 1}
	\end{align}
			
	%hawkings paper [2]

	%\clearpage
	
	%\subsection{Entropy and thermodynamics}
	\subsection{Entropy and thermodynamics \checkmark}
	If a black hole has a temperature and an energy, it must also have an entropy. So let's remember the inner energy in statistical mechanics\footnote{$\diff E = 
	T \diff S - p \diff V + \mu \diff N$} and derive it to:
		\begin{equation}
			\frac{\diff S}{\diff E} = \left. \frac{1}{T} \right|_{V,N=const.}
		\end{equation}
	For the black hole we have $T = \frac{1}{8 \pi G M}$ and $M=E$. If we assume that $S(E=0) = 0$, we can write:
		\begin{align}
			\frac{\diff S}{\diff M} &= 8 \pi G M \nonumber \\
			&\Leftrightarrow \int_0^S \diff S' = \int_0^M 8 \pi M' \diff M' \nonumber\\
			&\Leftrightarrow S = 4 \pi G M^2 & & \Bigm| r_s = 2GM \nonumber\\
			&\Leftrightarrow S = \frac{r_s^2 \pi}{G}  & & \Bigm| A=4 \pi r_s^2 \text{~and~} l_p = \sqrt{8 \pi G} \nonumber\\
			&\Leftrightarrow~
			S = \frac{A}{4G} = 2 \pi \frac{A}{l_p} \label{entropy}
		\end{align}
	For a black hole of the mass of our sun, this entropy would be $10^{78}$ which is enormous! If we take the sun like it is, the entropy would ``just'' be $10^{60}$. 
	
	Historical the entropy of a black hole was discovered before its temperature. With help of classical general relativity, we can see that the area of an event horizon of a black hole never decreases which looks quite like the second law of thermodynamics. Together with certain formal definition of the entropy, where it is proportional to the horizon area and a temperature indirect proportional to the Schwarzschild radius, the first law of thermodynamics with $\diff M = T \diff S$ is satisfied, too. 
	
	Jacob Bekenstein was holding out that this entropy should be that kind of statistical entropy of a black hole, that counts the number of ways it could have formed itself. 
	In a thought experiement he was throwing some systems with own entropy into a black hole and discovered that the interior entropy was growing faster, than the exterior entropy was sinking because of the systems loss. 
	So this means, that the entropy must be given by some constant proportional to the horizon area in Planck units.
	Bekenstein called this the \textit{Genereralized Second Law}.
	
	As Hawking published his paper about the temperature of a black hole, Bekensteins theory strongly reinforced. This is why the entropy of a black hole is often called \textbf{Bekenstein-Hawking entropy}. 
	
	In \textit{string theory} this idea of an entropy counting microstates is strong supported. For example in many situations where we count the states of a long vibrating string we can see how big the entropy of a black hole is. In some supersymmetric cases it is even possible to compute the $\frac{1}{4}$ in equation \eqref{entropy}.
	
	\subsection{What happens to the information while evaporation? N}
	Steven Hawking said in his paper\marginpar{[2]}, it is inconsistent with quantum mechanics, that the black hole's entropy counts the number of ways it could have been formed which most people would think in the first place. 
	
	The idea behind these thoughts is that the outgoing radiation of a black hole is completely independent of details of the initial state of photons. 
	In explicit we make a diagonal density matrix
		\begin{equation} \label{density matrix informprobl}
			\rho \propto \bigotimes_{\omega, l, m} \left(
			\sum_n \ket{n} \bra{n}_{\omega, l, m} P_{abs} (\omega, l) e^{-\beta \omega n}
			\right)
		\end{equation}
	which leads to the emission rate of \eqref{energy_flux}
		
	This should remind you of the Rindler result which means that this reduced density matrix for the right or left Rindler wedge is just a thermal density matrix. But back then, we did not calculate in the gravity, so there will be catastrophic consequences once we turn on the gravity again.
		
	Now, if a black hole was orginally formed in some pure state $\ket{\psi}$, its outside radiation field becomes more and more mixed as we move forward in time. Because we are normally only looking at the late radiation outside of the black hole, this doesn't seem problematic. 
		
	While the black hole evaporates and becomes smaller, its entanglement entropy is always increasing as seen in \eqref{density matrix informprobl}. The problem is, its size decreases until it is Planckian\footnote{The planck scale begins at the planck length $l_p = \sqrt{\frac{\hbar G}{c^3}} \approx 1,6 \cdot 10^{-35}\unit{m}$, while a proton is about the size of $8,4 \cdot 10^{-16}\unit{m}$} and in those kinds of systems our common physics can't help us any more.
	
	What happens to the entropy now? One of two things must happen:
		\begin{enumerate}[(1)]
			\item The evaporation stops at Planck size. This rest is also called ``remnant'' and its entanglement entropy must be enormously big, bigger than that of a black hole with a comparable mass. \label{evap. planck size}
			\item The black hole finishes the evaporation till there is nothing left. The law of energy conservation prohibits that the last boost of photons contains enough entanglement entropy for reproducing the initial state. But if information can not get lost, we would have to violate the quantum mechanics here.
			So in the end we would have a mixed state with an entropy comparable to the one of the initial horizons entropy. \label{evap. finished}
		\end{enumerate}
	The option number \eqref{evap. planck size} is in fact possible, but it would mean that there are objects with an infinit amount of states below any finite energy.
	Also if a black hole can form out of photons and gravitons why should it not be possible for it to disapear entirely back into photons and gravitons.
	
	Option \eqref{evap. finished} in contrast seems to be the better choice, but it also means that black holes can destroy information. So one must admitt that gravity and quantum mechanics are inconstistent, they have no theory in common.
	
	Let's have a look at an other option, which is nearly similar to option number \eqref{evap. finished}:
		\begin{enumerate}[(3)]
			\item While evaporating, the information is hidden in some entanglement between the Hawking photons blasted by the black hole (or the rest of it).
			In the end we have a pure state of the radiation field instead of a mixed state like in \eqref{evap. finished}. This is just possible if we don't look at too many photons at once, because in complicated states any small subsystem looks thermal, so it can justify \eqref{density matrix informprobl}. 		
		\end{enumerate}				
		

\newpage
%\thispagestyle{empty}
\printbibliography
\end{document}

%%% Local Variables: ***
%%% TeX-master: "main.tex" ***
%%% End: ***
